% \iffalse meta-comment
%<*internal>
\def\nameofplainTeX{plain}
\ifx\fmtname\nameofplainTeX\else
  \expandafter\begingroup
\fi
%</internal>
%<*install>
\input l3docstrip.tex
\keepsilent
\askforoverwritefalse
\nopreamble\nopostamble
\usedir{tex/latex/scontents}
\generate{%
  \file{\jobname.sty}{\from{\jobname.dtx}{package}}%
}%
%</install>
%<install>\endbatchfile
%<*internal>
\usedir{source/latex/scontents}
\generate{
  \file{\jobname.ins}{\from{\jobname.dtx}{install}}
}
\ifx\fmtname\nameofplainTeX
  \expandafter\endbatchfile
\else
  \expandafter\endgroup
\fi
%</internal>
%<*driver>
\documentclass[full]{l3doc}
\usepackage[top=0.5in,bottom=0.5in,left=2.0in,right=1in,footskip=0.2in,headsep=10pt]{geometry}
\usepackage[osf,mono=false,scale=0.95,llscaled=0.95]{libertine}
\usepackage{unicode-math}
\setmathfont[Scale = 0.95]{latinmodern-math.otf}
\usepackage[osf,scale=0.80,semibold]{sourcecodepro}
\usepackage{fontawesome5}
\newfontfamily\lmmitalic{lmmono10-italic.otf}[
   Scale             = 0.95,%
   Extension         = .otf,%
   ItalicFont        = lmmono10-italic,%
   SmallCapsFont     = lmmonocaps10-oblique,%
   SlantedFont       = lmmonoslant10-regular,
   ]
\newfontfamily\fetamontotf{ffmw10.otf}[
   Scale             = 0.95,%
   RawFeature        = {+latn,+rand,+kern,+size},%
   ]
\usepackage[svgnames]{xcolor}
\usepackage[sf,bf,compact,medium,pagestyles]{titlesec}
\usepackage{hyperxmp,lastpage,imakeidx,microtype,attachfile2}
\usepackage{adjustbox,multicol,listings,accsupp,titletoc}
\usepackage{scontents} % main
\usepackage[contents]{colordoc}

% Patching colordoc.sty to work with l3doc.cls
\ExplSyntaxOn
\cs_new_eq:cN { liii@xmacro@code } \__codedoc_xmacro_code:n
\ExplSyntaxOff
\begingroup
\makeatletter
\catcode`\[\@ne\catcode`\]\tw@
\@makebracesactive
\gdef{[\@openingbrace[\char'173]]
\gdef}[\@closingbrace[\char'175]]
\catcode`\|\z@\catcode`\%12
\catcode`\ \active\catcode`\\\active
|gdef|xmacro@code#1%    \end{macrocode}[|liii@xmacro@code[#1]|end[macrocode]]
|catcode`| 12|gdef|sxmacro@code#1%    \end{macrocode*}[|liii@xmacro@code[#1]|end[macrocode*]]
|endgroup

\EnableCrossrefs
\PageIndex
\CodelineIndex
\EnableDocumentation
\EnableImplementation
\DoNotIndex{\ , \1,\^}
\expandafter\DoNotIndex\expandafter{\string\{}
\expandafter\DoNotIndex\expandafter{\string\}}
\expandafter\DoNotIndex\expandafter{\string\begin}
\newcommand{\HP}[1]{\emph{\hyperpage{#1}}\normalsize}
\def\MYSortIndex#1#2{\index[userdoc]{#1\actualchar#2|HP}}

\indexsetup{level=\section,firstpagestyle=myheader}
\makeindex[name=userdoc,options=-s gind.ist,columnsep=15pt,title={Index of Documentation}]
\makeindex[options=-s gind.ist,columnsep=15pt,title={Index of Implementation}]
\setlength{\parindent}{0pt}

%^^A------------------- ONLY FOR DOCUMENTATION ------------------------%%
%^^A Colors for options
\definecolor{optcolor}{rgb}{0.281,0.275,0.485}
\definecolor{mypkgcolor}{RGB}{0,128,128}%
%^^A Custom \meta[...]{...}, \marg[...]{...} and \oarg[...]{...} with color
\ExplSyntaxOn
%^^A user level commands
\NewDocumentCommand{\mymeta}{O{}m}
  {
   \userdoc_meta_generic:Nnn \userdoc_meta:n { #1 } { #2 }
  }
\NewDocumentCommand{\mymarg}{O{}m}
  {
   \userdoc_meta_generic:Nnn \userdoc_marg:n { #1 } { #2 }
  }
\NewDocumentCommand{\myoarg}{O{}m}
  {
   \userdoc_meta_generic:Nnn \userdoc_oarg:n { #1 } { #2 }
  }
%^^A variables and keys
\tl_new:N \l_userdoc_meta_font_tl

\keys_define:nn { userdoc / mymeta }
  {
   type .choice:,
   type / tt .code:n = \tl_set:Nn \l_userdoc_meta_font_tl { \ttfamily },
   type / rm .code:n = \tl_set:Nn \l_userdoc_meta_font_tl { \rmfamily },
   type .initial:n = tt,
   cf .tl_set:N = \l_userdoc_meta_color_tl,
   cf .initial:n = black,
   ac .tl_set:N = \l_userdoc_meta_anglecolor_tl,
   ac .initial:n = black,
   sbc .tl_set:N = \l_userdoc_meta_brackcolor_tl,
   sbc .initial:n = black,
   cbc .tl_set:N = \l_userdoc_meta_bracecolor_tl,
   cbc .initial:n = black,
  }
%^^A internal commands
\cs_new_protected:Npn \userdoc_meta_generic:Nnn #1 #2 #3
  {
   \group_begin:
   \keys_set:nn { userdoc / mymeta } { #2 }
   \color{ \l_userdoc_meta_color_tl }
   \l_userdoc_meta_font_tl
   #1 { #3 } % #1 is \userdoc_meta:n, \userdoc_marg:n or \userdoc_oarg:n
   \group_end:
  }
\cs_new_protected:Npn \userdoc_meta:n #1
  {
   \userdoc_meta_angle:n { \textlangle }
   \userdoc_meta_meta:n { #1 }
   \userdoc_meta_angle:n { \textrangle }
  }
\cs_new_protected:Npn \userdoc_marg:n #1
  {
   \userdoc_meta_brace:n { \textbraceleft }
   \userdoc_meta:n { #1 }
   \userdoc_meta_brace:n { \textbraceright }
  }
\cs_new_protected:Npn \userdoc_oarg:n #1
  {
   \userdoc_meta_brack:n { [ }
   \userdoc_meta:n { #1 }
   \userdoc_meta_brack:n { ] }
  }
\cs_new_protected:Npn \userdoc_meta_meta:n #1
  {
   \textnormal{\textit{#1}}
  }
\cs_new_protected:Npn \userdoc_meta_angle:n #1
  {
   \group_begin:
   \fontfamily{lmr}\selectfont
   \textcolor{\l_userdoc_meta_anglecolor_tl}{#1}
   \group_end:
  }
\cs_new_protected:Npn \userdoc_meta_brace:n #1
  {
   \group_begin:
   \color{\l_userdoc_meta_bracecolor_tl}
    #1
   \group_end:
  }
\cs_new_protected:Npn \userdoc_meta_brack:n #1
  {
   \textcolor{\l_userdoc_meta_brackcolor_tl}{#1}
  }

%^^A \envexamp{m}
\newsavebox{\boxexaenv}
\NewDocumentCommand{\envexamp}{m}
  {
   \begin{lrbox}{\boxexaenv}%
    \begin{minipage}[t]{\marginparwidth}%
     \raggedright\ttfamily\small
     \textcolor{gray}{\textbackslash begin\{\textcolor{mypkgcolor}{\bfseries{#1}}\}\myoarg
     [type=tt,sbc=gray,ac=NavyBlue,cf=optcolor]{key=val}}\par%
     \hspace{0.5cm}\mymeta[ac=gray,cf=gray]{env ~ contents}\par%
     \textcolor{gray}{\textbackslash end\{{\textcolor{mypkgcolor}{\bfseries{#1}}}\}}\par
    \end{minipage}%
   \end{lrbox}%
   \usebox{\boxexaenv}
  }

%^^A \cmdexamp{s m o m o}
\DeclareDocumentCommand{\cmdexamp}{s m o m o}
  {
  \group_begin:
  \small\ttfamily
  \textcolor{mypkgcolor}{\textbackslash#2}
  \IfBooleanTF{#1}{ \textcolor{red}{*} }{ \hphantom{*} }
  \IfValueT{#3}{ \myoarg[type=tt,sbc=gray,ac=NavyBlue,cf=optcolor]{#3} }
  \IfValueTF{#5}
    {
     \mymeta[ac=OrangeRed,type=tt,cf=MediumOrchid]{#5}%
     \mymeta[type=tt,cbc=OrangeRed,ac=NavyBlue,cf=MediumOrchid]{#4}
     \mymeta[ac=OrangeRed,type=tt,cf=MediumOrchid]{#5}%
    }
    { \mymarg[type=tt,cbc=OrangeRed,ac=NavyBlue,cf=MediumOrchid]{#4} }
  \par
  \group_end:
  \MYSortIndex{Commands}{Commands ~ provide  ~ by  ~ \textcolor{gray}{scontents}>\texttt{\textbackslash#2}}%
  }

%^^A \keyexamp{mmm}
\DeclareDocumentCommand{\keyexamp}{ m m m }
  {
  \par
  \adjustbox{outer=-\marginparsep}{\textcolor{mypkgcolor}{\small\ttfamily{#1}}}
  \textcolor{gray}{\,\bfseries\texttt{=}}\,{}
  \mymarg[type=tt,cbc=OrangeRed,ac=NavyBlue,cf=MediumOrchid]{\small{#2}}
  \hfill\textcolor{gray}{\small\textsf{(default:  ~ \texttt{#3})}}
  \par
  \MYSortIndex{Keys}{Keys>\texttt{#1}}%
  }

%^^A \mypkg{sm}
\NewDocumentCommand{\mypkg}{sm}
  {
   \IfBooleanTF{#1}
    {
     \normalsize{\sffamily\textcolor{mypkgcolor}{s}\textcolor{OrangeRed}{content}\textcolor{mypkgcolor}{s}}
     \MYSortIndex{packages}{Packages>\texttt{#2}}
    }
    {
     \textcolor{gray}{\textsf{#2}}
     \MYSortIndex{packages}{Packages>\texttt{#2}}%
    }
  }

%^^A \myenv{sm}
\DeclareDocumentCommand{\myenv}{sm}
  {
   \IfBooleanTF{#1}
    {
     \textcolor{mypkgcolor}{\ttfamily{#2}}%
     \MYSortIndex{environment}{Environment ~ provide ~ by ~ \textcolor{gray}{scontents}:>\texttt{#2}}
    }
    {
     \textcolor{gray}{\ttfamily{#2}}%
     \MYSortIndex{environment}{Environments>\texttt{#2}}
    }
  }

%^^A \ics{sm}
\DeclareDocumentCommand{\ics}{sm}
  {
    \IfBooleanTF{#1}
      {
        \textcolor{mypkgcolor}{\ttfamily\textbackslash{#2}}
        \MYSortIndex{Commands}{Commands ~ provide  ~ by  ~ \textcolor{gray}{scontents}>\texttt{\textbackslash#2}}
      }
      {
        \textcolor{gray}{\ttfamily\textbackslash{#2}}
        \MYSortIndex{#2}{\texttt{\textbackslash#2}}
      }
  }
\ExplSyntaxOff

%^^A email https://tex.stackexchange.com/a/663
\catcode`\_=11\relax%
\newcommand\email[1]{\_email #1\q_nil}%
\def\_email#1@#2\q_nil{%
  \href{mailto:#1@#2}{{\emailfont #1\emailampersat #2}}%
}%
\newcommand\emailfont{\sffamily}%
\newcommand\emailampersat{{\color{optcolor}\footnotesize @}}%
\catcode`\_=8\relax%

\makeatletter
%^^A Logo whit fetamont font for title
\newsavebox{\logobox}
\savebox{\logobox}{%
  \normalsize\fetamontotf{\textcolor{mypkgcolor}{s}\textcolor{OrangeRed}{content}\textcolor{mypkgcolor}{s}}}
\newcommand{\mylogo}{%
  \settoheight{\@tempdima}{L}%
  \resizebox{!}{\@tempdima}{\usebox{\logobox}}%
 }
\makeatother

%^^A Don't copy numbers in code example
\newcommand*{\noaccsupp}[1]{\BeginAccSupp{ActualText={}}#1\EndAccSupp{}}

%^^A Create a language for documentation
\lstdefinelanguage{scontents-doc}{
    texcsstyle=*,%
    escapechar=`,%
    showstringspaces=false,%
    extendedchars=true, %
    stringstyle = {\color{red}},%
    basicstyle=\ttfamily\small,%
% comments
    morecomment=[l]{\%},%
    commentstyle=\lmmitalic\small\itshape\color{lightgray},%
% Important words 1
    keywordstyle=[1]{\bfseries\color{NavyBlue}},%
    keywords=[1]{begin,end,documentclass},%
% Other words 2
    keywordstyle=[2]{\color{blue!75}},%
    keywords=[2]{usepackage,section},%
% Other words 3
    keywordstyle=[3]{\color{optcolor!85}},%
    keywords=[3]{document,article},%
% Reserved words 4(inputfile options)
    keywordstyle=[4]{\bfseries\color{mypkgcolor}},%
    keywords=[4]{scontents,Scontents,getstored,typestored,verbatimsc,endverbatimsc,countsc},%
% Reserved in orange
    keywordstyle=[5]{\color{OrangeRed}},%
    keywords=[5]{makeatletter,makeatother,foreach,DefineVerbatimEnvironment,lstnewenvironment,newminted},%
% Reserved in orange
    keywordstyle=[6]{\color{red}},%
    keywords=[6]{verb,myverb,store-cmd,store-env},%
% literateee
    literate=*{\{}{{\bfseries\textcolor{gray}{\{}}}{1}
              {\}}{{\bfseries\textcolor{gray}{\}}}}{1}
              {[}{{\bfseries\textcolor{optcolor}{[}}}{1}
              {]}{{\bfseries\textcolor{optcolor}{]}}}{1}
              {,}{{\textcolor{gray}{,}}}{1}
              {\$}{{\textcolor{blue}{\$}}}{1}
              {*}{{\bfseries\textcolor{red}{*}}}{1}
              {@}{{\textcolor{red}{@}}}{1}
              {=}{{\textcolor{red}{=}}}{1},%
}[keywords,tex,comments,strings]% end languaje

%^^A \begin{examplecode}[key=val]...\end{examplecode}
\lstnewenvironment{examplecode}[1][]{%
\lstset{
    language=scontents-doc,%
    stringstyle = {\color{red}},%
    basicstyle=\ttfamily\small,%
    numbersep=1em,%
    numberstyle=\tiny\color{lightgray}\noaccsupp,%
    rulecolor=\color{gray!50},%
    framesep=\fboxsep,%
    framerule=\fboxrule,%
    xleftmargin=\dimexpr\fboxsep+\fboxrule\relax,%
    xrightmargin=\dimexpr\fboxsep+\fboxrule\relax,%
           #1,%
    }% close lstset
}{}% close examplecode

%^^A \lstinline[style=inline]|...|
\lstdefinestyle{inline}
  {
   language=scontents-doc,%
   numbersep=1em,%
   numberstyle=\tiny\color{lightgray}\noaccsupp,%
   basicstyle=\ttfamily\small\color{gray},%
   escapechar=`,%
   upquote=true,%
   literate=*{\%}{{\bfseries\textcolor{gray}{\%}}}{1}
  }

%^^A Set default style
\lstset{style=inline}

%^^A Get file info
\GetFileInfo{\jobname.sty}

%^^A Config hyperref
\hypersetup{
   linkcolor          = blue!50,
   citecolor          = red!50,%
   urlcolor           = magenta,%
   colorlinks         = true,%
   linktoc            = all,%
   pdftitle           = {.:: The scontents package --- \fileinfo{} ::.},%
   pdfauthor          = {Pablo González L},
   pdfsubject         = {Documentation for \fileversion{} [\filedate] },%
   pdfcopyright       = {\textcopyright 2019 by Pablo González},
   pdfcontacturl      = {https://github.com/pablgonz/scontents},
   pdfkeywords        = {filecontents, filecontentsdef, xparse, expl3, l3seq, Store contents},
   pdfstartview       = {FitH},%
   bookmarksopenlevel = 1,%
  }

%^^A Configuration titleps and titlesec
\settitlemarks{section}
\renewpagestyle{plain}[\color{gray}\small\sffamily]{
\setfoot{}{}{\thepage/\pageref{LastPage}}}

\newpagestyle{myheader}[\color{gray}\small\sffamily]{
\renewcommand\makeheadrule{\color{gray}\rule[0.45\baselineskip]{\linewidth}{0.4pt}}
\setfoot{\scalebox{0.8}{\mylogo}\space\textcopyright\space 2019 by Pablo González}
        {}
        {\thepage/\pageref{LastPage}}
\sethead{\raisebox{0.75\baselineskip}{Documentation for \fileversion\space[\filedate]}}
        {}
        {\raisebox{0.75\baselineskip}{\scshape\small\S.\thesection\space\sectiontitle}}
}

\setlength{\headheight}{21pt}%

\titlecontents{section}[0mm]{}%
    {\bfseries\contentspush{\makebox[5mm][l]{\thecontentslabel\hfill}}}%
    {\hspace*{-5mm}}% numberless
    {\hspace{0.25em}\titlerule*[6pt]{.}\contentspage}%
\titlecontents{subsection}[5mm]{}%
    {\contentspush{\makebox[6mm][l]{\thecontentslabel\hfill}}}
    {\hspace*{-11mm}}% numberless
    {\hspace{0.25em}\titlerule*[6pt]{.}\contentspage}%

%^^A Table of contents
\makeatletter
\renewcommand\tableofcontents{%
\begingroup%
\section*{\contentsname\quad{\color{gray}\leaders\hrule height 5pt depth -4.4pt\hfill}%
  \@mkboth{%
    \MakeUppercase\contentsname}{\MakeUppercase\contentsname}}%
\vspace*{-14pt}
\setlength{\columnsep}{10pt}%
 \begin{multicols}{2}%
    \@starttoc{toc}%
\end{multicols}%
\vspace*{-3pt}{\color{gray}\hrule height 0.6pt}%
\vspace*{5pt}
\endgroup
}
\makeatother
\begin{document}
  \DocInput{\jobname.dtx}
\end{document}
%</driver>
% \fi
%
% \begin{documentation}
%
% \title{
%    \scalebox{1.025}{\mylogo}\\[2pt]
%    \Large
%    \textsf{Stores \hologo{LaTeX} contents}\\[3pt]
%    \fileversion{} --- \filedate\thanks{
%    This file describes a documentation for \fileversion, last revised
%    \filedate.}\\[25pt]
%    \author{
%    \large
%    \raisebox{-1pt}{\textcopyright}{}2019 by Pablo González
%    \thanks{E-mail: \textsf{\guillemotleft}\email{pablgonz@educarchile.cl}\textsf{\guillemotright}}%
%    }
% \small
% \textsc{ctan}: \url{http://www.ctan.org/pkg/scontents}\\
% \textsc{git}: \url{https://github.com/pablgonz/scontents}
% \vspace*{-2cm}
% }
% \date{}
% \maketitle
%
% \begin{abstract}
% The \mypkg*{scontents} package stores valid \hologo{LaTeX} code in memory (sequences) using
% the \mypkg{l3seq} module of \mypkg{expl3}. The stored content (including
% \emph{verbatim}) can be used as many times as desired in the document,
% additionally can be written to external files if desired.
% \end{abstract}
%
% \tableofcontents
%
% \setlength{\parskip}{3pt}
%
% \section{Motivation and Acknowledgments}
%
% In \hologo{LaTeX} there is no direct way to record content for later use, although
% you can do this using \verb|\macros|, recording \meta{verbatim content}  is a
% problem, usually you can avoid this by creating external files or boxes. The
% general idea of this package is to try to imitate this implementation
% \emph{buffers} that has \hologo{ConTeXt} which allows you to save content in
% memory, including \emph{verbatim}, to be used later. The package
% \mypkg{filecontentsdef} solves this problem and since \mypkg{expl3} has an
% excellent way to manage data, I decided to combine the best of both.
%
% This package would not be possible without the great work of \textsc{Jean
% Fran\c{c}ois Burnol} who was kind enough to take my requirements into account
% and add the \myenv{filecontentsdefmacro} environment. Also a special thanks to
% Phelype Oleinik who has collaborated and adapted a large part of the code and
% all  \hologo{LaTeX3} team for their great work and to the different members
% of the \href{https://tex.stackexchange.com}{TeX-SX} community who have provided
% great answers and ideas. Here a note of the main ones:
%
% \begin{enumerate}[nosep]
% \item \href{https://tex.stackexchange.com/q/45946/7832}{Stack datastructure using LaTeX}
%
% \item \href{https://tex.stackexchange.com/q/5338/7832}{LaTeX equivalent of ConTeXt buffers}
%
% \item \href{https://tex.stackexchange.com/q/215563/7832}{Storing an array of strings in a command}
%
% \item \href{https://tex.stackexchange.com/q/184503/7832}{Collecting contents of environment and store them for later retrieval}
%
% \item \href{https://tex.stackexchange.com/q/373647/7832}{Collect contents of an environment (that contains verbatim content)}
% \end{enumerate}
%
% \section{License and Requirements}
% \label{sec:licence}
%
% Permission is granted to copy, distribute and/or modify this software under
% the terms of the LaTeX Project Public License (lppl), version 1.3 or later
% (\url{http://www.latex-project.org/lppl.txt}). The software has the status
% \enquote{maintained}.
%
% The \mypkg*{scontents} package loads \mypkg{xparse},
% \mypkg{filecontentsdef} and \mypkg{l3keys2e}. This package can be used
% with |xelatex|, |lualatex|, |pdflatex| and the classical workflow
% |latex|-|dvips|-|ps2pdf|.
%
% \thispagestyle{plain}
%
% \newpage
%
% \pagestyle{myheader}
%
% \section{The \texttt{scontents} package}
%
% \subsection{Description}
%
% The \mypkg*{scontents} package encapsulates the \mypkg{filecontentsdef}
% package of \textsc{Jean Fran\c{c}ois Burnol} which allows you to save the
% content in a |\macro| and save it in external files, adding a user interface
% style \myoarg{key = val} along with the ability to save content in sequences
% for later use in different parts of the document.
%
% \subsection{Loading package}
% \label{sec:loadoptpkg}
%
% The package is loaded in the usual way:
%
% \iffalse
%<*example>
% \fi
\begin{examplecode}[frame=single]
\usepackage{scontents}
\end{examplecode}
% \iffalse
%</example>
% \fi
% or
% \iffalse
%<*example>
% \fi
\begin{examplecode}[frame=single]
\usepackage`\myoarg[type=tt,cbc=OrangeRed,ac=NavyBlue,cf=optcolor]{key=val}`{scontents}
\end{examplecode}
% \iffalse
%</example>
% \fi
%
% \subsection{Configuration of the options}
% \label{sec:confopt}
%
% Most of the options can be passed directly to the package or can be
% configured by means of the command \ics*{setupsc}.
%
% \vspace*{-10pt}
%
% \begin{function}{\setupsc}
%   \begin{syntax}
%       \cmdexamp{setupsc}{key=val}
%   \end{syntax}
%
% The command \ics*{setupsc} configures the options in a global way,
% for \ics*{Scontents}, \ics*{Scontents*} and environment \myenv*{scontents}.
% it can be used both in the preamble and in the body of the document
% as many times as desired.
% \end{function}
%
% \subsection{Options Overview}
% \label{sec:optover}
%
% \newcommand*{\xmark}{\textcolor{red}{✘}}%
% \newcommand*{\cmark}{\textcolor{green}{✔}}%
%
% Summary table of available options.
%
% \begin{tabular}{cccccc}
% \toprule
%  \texttt{key}       & package & \ics{setupsc} & \myenv*{scontents}  & \ics*{Scontents} & \ics*{Scontents*}\\
% \midrule
%  \texttt{store-env} & \cmark  & \cmark        &  \cmark             & \xmark          & \xmark          \\
%  \texttt{store-cmd} & \cmark  & \cmark        &  \xmark             & \cmark          & \cmark          \\
%  \texttt{print-env} & \cmark  & \cmark        &  \cmark             & \xmark          & \xmark          \\
%  \texttt{print-cmd} & \cmark  & \cmark        &  \xmark             & \cmark          & \cmark          \\
%  \texttt{print-all} & \cmark  & \cmark        &  \cmark             & \xmark          & \xmark          \\
%  \texttt{write-env} & \xmark  & \xmark        &  \cmark             & \xmark          & \xmark          \\
%  \texttt{write-out} & \xmark  & \xmark        &  \cmark             & \xmark          & \xmark          \\
% \bottomrule
% \end{tabular}
%
% \section{User interface}
%
% The \emph{user interface} provided by this package consists in \myenv*{scontents} environment,
% \ics*{Scontents} and \ics*{Scontents*} commands to stored contents and \ics*{getstored} command
% to get the \meta{stored content} along with other utilities described in this documentation.
%
% \subsection{The environment \env{scontents}}
% \label{sec:envscontents}
%
% \vspace*{-10pt}
%
% \begin{function}{scontents}
%   \begin{syntax}
%   \envexamp{scontents}
%   \end{syntax}
%
% The \myenv*{scontents} environment encapsulates the \myenv{filecontentsdef*} and
% \myenv{filecontentsdefmacro} environments provided by the \mypkg{filecontentsdef}
% package. This allows you to record content including verbatim for later reuse.
% \end{function}
%
% Some considerations to keep in mind:
%
% \begin{enumerate}[nosep]
% \item The environment cannot be nested.
% \item Both \lstinline[style=inline]|\begin| and \lstinline[style=inline]|\end|
%  must be on different lines.
% \item The \myoarg{key=val} options must be passed on one line right after
%  starting the environment.
% \item The content of the environment is treated in the same way as
% \myenv{filecontents*} environment.
% \item If you don't want the extra space added by \hologo{TeX}, you should use
% \ics{relax} or \lstinline[style=inline]|%| at the end of environment.
% \end{enumerate}
%
% For more technical information about the environment it is better to read
% the documentation of the \mypkg{filecontentsdef} package.
%
% \subsection*{Options for environment}
%
% The environment options can be configured globally using option
% in package or the \ics*{setupsc} command and locally
% using \myoarg{key=val} in the environment.
%
% \medskip
%
% \keyexamp{store-env}{seq name}{contents}
% The name of the sequence in which the content recorded by the environment
% was stored.
%
% \medskip
%
% \keyexamp{print-env}{true\textbar false}{false}
% It will show the current content of the environment.
%
% \medskip
%
% \keyexamp{write-env}{file.ext}{not used}
% In addition to storing the content of the environment will write this in
% an external file.
%
% \medskip
%
% \keyexamp{write-out}{file.ext}{not used}
% It will write the contents of the environment in an external file, but,
% it will not store the contents of this one. It is analogous to the
% \myenv{filecontents*} environment.
%
% \subsection{The command \cs{Scontents}}
% \label{sec:Scontents}
%
% The command to store content directly in memory, the star version allows to save
%  \meta{verbatim} contents.
%
% \vspace*{-10pt}
%
% \begin{function}{\Scontents}
%   \begin{syntax}
%     \cmdexamp{Scontents}[key=val]{argument}
%     \cmdexamp*{Scontents}[key=val]{argument}
%     \cmdexamp*{Scontents}[key=val]{argument}[del]
%   \end{syntax}
% \end{function}
%
% The \ics*{Scontents} command reads the \meta{argument} in standard mode.
% It is not possible to pass environments such as \meta{verbatim}, but it is
% possible to use the implementation of \ics{Verb} provided by the
% \mypkg{fvextra} package for contents on one line and \ics{lstinline} from
% \mypkg{listings} package, but it is preferable to use the starred version.
%
% It can be used anywhere in the document and cannot be used as an \meta{argument}
% for another command.
%
% The \ics*{Scontents*} command reads the \meta{argument} under  verbatim
% category code regimen. If its first delimiter is a brace, it will be assumed that
% the \meta{argument} is nested into braces. Otherwise it will be assumed that
% the ending of that argument is delimited by that first delimiter-like the
% argument of \ics{verb}. Some considerations to keep in mind:
%
% \begin{enumerate}[nosep]
% \item Blank lines are preserved.
% \item The command cannot be used as an argument for another command.
% \item If you don't want the extra space added by \hologo{TeX}, you should use
% \ics{relax} or \lstinline[style=inline]|%| at the end.
% \end{enumerate}
%
% \subsection*{Options for command}
%
% The command options (including star version) can be configured globally
% using option in package or the \ics*{setupsc} command and locally
% using \myoarg{key=val}.
% \medskip
%
% \keyexamp{store-cmd}{seq name}{contents}
% The name of the sequence in which the content recorded by \ics*{Scontents}
% was stored.
%
% \medskip
%
% \keyexamp{print-cmd}{true\textbar false}{false}
% It will show the current content of \ics*{Scontents}.
%
% \subsection{The command \cs{getstored}}
% \label{sec:getstored}
%
% \vspace*{-10pt}
%
% \begin{function}{\getstored}
%   \begin{syntax}
%      \cmdexamp{getstored}[index]{seq name}
%   \end{syntax}
% \end{function}
%
% The command \ics*{getstored} gets the content stored in \meta{seq name}
% according to the index in which it was stored. The command is robust and expandable
% and can be used as an \meta{argument} for another command. If the optional argument
% is not passed it defaults to the last element saved in the \meta{seq name}.
%
% \subsection{The command \cs{typestored}}
% \label{sec:typestored}
%
% \vspace*{-10pt}
%
% \begin{function}{\typestored}
%   \begin{syntax}
%      \cmdexamp{typestored}[index]{seq name}
%      \cmdexamp*{typestored}[index]{seq name}
%   \end{syntax}
% \end{function}
%
% The command \ics*{typestored} shows the content stored in \meta{seq name} in
% \emph{verbatim} mode. Internally places the content into the \myenv*{verbatimsc} environment.
%
% The command \ics*{typestored*} must be used for content stored by \ics*{Scontents*} command.
% Internally places the content into the \myenv*{verbatimsc} environment.
%
% If the optional argument is not passed it defaults to the
% last element saved in the \meta{seq name}.
%
% \subsection{The environment \env{verbatimsc}}
%
% \vspace*{-10pt}
%
% \begin{function}{verbatimsc}
% Internal environment used by \cs{typestored} and \cs{typestored*}  to display
% \meta{verbatim style} contents.
% \end{function}
%
% One consideration to keep in mind is that this is a \emph{representation} of the content
% in a \meta{verbatim} environment and not a real \meta{verbatim} environment, the line ends
% are not respected. The \myenv*{verbatimsc} environment can be in the following ways:
%
% Using the package \mypkg{fancyvrb}:
% \iffalse
%<*example>
% \fi
\begin{examplecode}[frame=single]
\makeatletter
\let\verbatimsc\@undefined
\let\endverbatimsc\@undefined
\makeatother
\DefineVerbatimEnvironment{verbatimsc}{Verbatim}{numbers=left}
\end{examplecode}
% \iffalse
%</example>
% \fi
% Using the package \mypkg{minted}:
% \iffalse
%<*example>
% \fi
\begin{examplecode}[frame=single]
\makeatletter
\let\verbatimsc\@undefined
\let\endverbatimsc\@undefined
\makeatother
\usepackage{minted}
\newminted{tex}{linenos}
\newenvironment{verbatimsc}{\VerbatimEnvironment\begin{texcode}}{\end{texcode}}
\end{examplecode}
% \iffalse
%</example>
% \fi
% Using the package \mypkg{listings}:
% \iffalse
%<*example>
% \fi
\begin{examplecode}[frame=single]
\makeatletter
\let\verbatimsc\@undefined
\let\endverbatimsc\@undefined
\makeatother
\usepackage{listings}
\lstnewenvironment{verbatimsc}
  {
   \lstset{
           basicstyle=\small\ttfamily,
           columns=fullflexible,
           language=[LaTeX]TeX,
           numbers=left,
           numberstyle=\tiny\color{gray},
           keywordstyle=\color{red}
          }
  }{}
\end{examplecode}
% \iffalse
%</example>
% \fi
%
% \section{Other commands provided}
%
% \subsection{The command \cs{meaningsc}}
% \label{sec:meaningsc}
%
% \vspace*{-10pt}
%
% \begin{function}{\meaningsc}
%   \begin{syntax}
%      \cmdexamp{meaningsc}[index]{seq name}
%   \end{syntax}
% The command \ics*{meaningsc} executes \ics{meaning} on the content stored
% in \meta{seq name}. If the optional argument is not passed it defaults to the
% last element saved in the \meta{seq name}.
% \end{function}
%
% \subsection{The command \cs{countsc}}
% \label{sec:countsc}
%
% \vspace*{-10pt}
%
% \begin{function}{\countsc}
%   \begin{syntax}
%      \cmdexamp{countsc}{seq name}
%   \end{syntax}
% The command \ics*{countsc} count a number of contents stored in \meta{seq name}.
% \end{function}
%
% \subsection{The command \cs{cleansc}}
% \label{sec:cleansc}
%
% \vspace*{-10pt}
%
% \begin{function}{\cleanseqsc}
%   \begin{syntax}
%      \cmdexamp{cleansc}{seq name}
%   \end{syntax}
% The command \ics*{cleansc} remove all contents stored in \meta{seq name}.
% \end{function}
%
% \section{Examples}
%
% These are some (adapted) examples that have served as inspiration for
% the creation of this package.
%
% \subsection{From answers package}
%
% \subsubsection*{Example 1}
%
% \begin{VerbatimOut}{scexamp1.ltx}
% \documentclass[12pt,a4paper]{article}
% \usepackage[store-cmd=solutions]{scontents}
% \usepackage{pgffor}
% \newtheorem{ex}{Exercise}
% \begin{document}
% \section{Problems}
% \begin{ex}
%   First exercise
%    \Scontents{
%       First solution.
%    }
% \end{ex}
% \begin{ex}
%   Second exercise
%   \Scontents{
%      Second solution.
%   }
% \end{ex}
% \section{Solutions}
% \foreach \i in {1,...,\countsc{solutions}} {
% \noindent\textbf{\i} \getstored[\i]{solutions}\par
% }
% \end{document}
% \end{VerbatimOut}
%
% Adaptation of example 1 (ansexam1) of the package \mypkg{answers}
% \textattachfile[color=0 0 1]{scexamp1.ltx}{\faFile*[regular]}.
% \lstinputlisting[language=scontents-doc,numbers=left]{scexamp1.ltx}
%
% \subsection{From filecontentsdef package}
%
% \subsubsection*{Example 2}
%
% \begin{VerbatimOut}{scexamp2.ltx}
% \documentclass{article}
% \usepackage[store-env=defexercise,store-cmd=defexercise]{scontents}
% \usepackage{pgffor}
% \pagestyle{empty}
% \begin{document}
% \Scontents{
% Prove that \[x^n+y^n=z^n\] is not solvable in positive integers if $n$ is at
% most $-3$.\par
% }
% \Scontents*{Refute the existence of black holes in less than $140$ characters.\relax}
% \begin{scontents}[write-env=\jobname-3.txt]
% \def\NSA{NSA}%
% Prove that factorization is easily done via probabilistic algorithms and
% advance evidence from knowledge of the names of its employees in the
% seventies that the \NSA\ has known that for 40 years.\par
% \end{scontents}
%
% \foreach \i in {1,...,3} {
% \begin{itemize}
% \item \getstored[\i]{defexercise}
% \end{itemize}}
%
% \section{\getstored[2]{defexercise}} % \getstored are robust :)
% \end{document}
% \end{VerbatimOut}
%
% Adaptation of example from package \mypkg{filecontentsdef}
% \textattachfile[color=0 0 1]{scexamp2.ltx}{\faFile*[regular]}.
% \lstinputlisting[language=scontents-doc,numbers=left]{scexamp2.ltx}
%
% \subsection{From TeX-SX}
%
% \subsubsection*{Example 3}
%
% \begin{VerbatimOut}{scexamp3.ltx}
% \documentclass{article}
% \usepackage[store-cmd=tikz]{scontents}
% \usepackage{tikz}
% \pagestyle{empty}
% \Scontents*{\matrix{ \node (a) {$a$} ; & \node (b) {$b$} ; \\ } ;}
% \Scontents*{\matrix[ampersand replacement=\&]
%    { \node (a) {$a$} ; \& \node (b) {$b$} ; \\ } ;}
% \Scontents*{\matrix{\node (a) {$a$} ; & \node (b) {$b$} ; \\ } ; }
% \begin{document}
% \section{tikzpicture}
% \begin{tikzpicture}
% \getstored[1]{tikz}
% \end{tikzpicture}
% \begin{tikzpicture}
% \getstored[2]{tikz}
% \end{tikzpicture}
% \begin{tikzpicture}
% \getstored[3]{tikz}
% \end{tikzpicture}
% \section{source}
% \foreach \i in {1,...,\countsc{tikz}}{
% \typestored*[\i]{tikz}}
% \end{document}
% \end{VerbatimOut}
% Adapted from \href{https://tex.stackexchange.com/q/5338/7832}{LaTeX equivalent of ConTeXt buffers}
% \textattachfile[color=0 0 1]{scexamp3.ltx}{\faFile*[regular]}.
% \lstinputlisting[language=scontents-doc,numbers=left]{scexamp3.ltx}
%
% \subsubsection*{Example 4}
%
% \begin{VerbatimOut}{scexamp4.ltx}
% \documentclass{article}
% \usepackage{scontents}
% \usepackage{pgffor}
% \pagestyle{empty}
% \begin{document}
% \begin{scontents}[store-env=a]
% Something for a
% \end{scontents}
%
% \begin{scontents}[store-env=a]
% Something for b
% \end{scontents}
%
% \begin{scontents}[store-env=a]
% Something with no label
% \end{scontents}
%
% \textbf{Let's print them}
%
% This is a: \getstored[1]{a}
%
% This is b: \getstored[2]{a}
%
% \textbf{Print all of them}
%
% \foreach \i in {1,...,\countsc{a}} {\getstored[\i]{a}\par}
% \end{document}
% \end{VerbatimOut}
%
% Adapted from \href{https://tex.stackexchange.com/q/184503/7832}{Collecting contents of environment and store them for later retrieval}
% \textattachfile[color=0 0 1]{scexamp4.ltx}{\faFile*[regular]}.
% \lstinputlisting[language=scontents-doc,numbers=left]{scexamp4.ltx}
%
% \subsubsection*{Example 5}
%
% \begin{VerbatimOut}{scexamp5.ltx}
% \documentclass{article}
% \usepackage{scontents}
% \pagestyle{empty}
% \setlength{\parindent}{0pt}
% \begin{document}
% \section{Problem stated the first time}
% \begin{scontents}[print-env=true,store-env=problem]
% This is normal text. \verb+This is from the verb command+. This is normal text.
% \verb*|This is from the verb* command|. This is normal text.
% \begin{verbatim}
% This is from the verbatim environment:
% &%{}
% \end{verbatim}
% \end{scontents}
% \section{Problem restated}
% \getstored[1]{problem}
% \section{Problem restated once more}
% \getstored[1]{problem}
% \end{document}
% \end{VerbatimOut}
%
% Adapted from \href{https://tex.stackexchange.com/q/373647/7832}{Collect contents of an environment (that contains verbatim content)}
% \textattachfile[color=0 0 1]{scexamp5.ltx}{\faFile*[regular]}.
% \lstinputlisting[language=scontents-doc,numbers=left]{scexamp5.ltx}
%
% \subsection{Customization of verbatimsc}
%
% \subsubsection*{Example 6}
%
% \begin{VerbatimOut}{scexamp6.ltx}
% \documentclass{article}
% \usepackage{scontents}
% \makeatletter
% \let\verbatimsc\@undefined
% \let\endverbatimsc\@undefined
% \makeatother
% \usepackage{fvextra}
% \usepackage{xcolor}
% \definecolor{mygray}{gray}{0.9}
% \usepackage{tcolorbox}
% \newenvironment{verbatimsc}%
% {\VerbatimEnvironment
%  \begin{tcolorbox}[colback=mygray, boxsep=0pt, arc=0pt, boxrule=0pt]
%  \begin{Verbatim}[fontsize=\scriptsize, breaklines, breakafter=*, breaksymbolsep=0.5em,
%    breakaftersymbolpre={\,\tiny\ensuremath{\rfloor}}]}%
% {\end{Verbatim}%
%  \end{tcolorbox}}
% \setlength{\parindent}{0pt}
% \pagestyle{empty}
% \begin{document}
%
% \section{Test \texttt{\textbackslash begin\{scontents\}} whit \texttt{fancyvrb}}
% Test \verb+\begin{scontents}+ \par
%
% \begin{scontents}
% Using \verb+scontents+ env no \verb+[key=val]+, save in seq \verb+contents+
% with index 1.
%
% Prove new \Verb*{ fancyvrb whit braces } and environment \verb+Verbatim*+
% \begin{verbatim}
%     verbatim  environment
% \end{verbatim}
% \end{scontents}
%
% \section{Test \texttt{\textbackslash Scontents} whit \texttt{fancyvrb}}
%
% \Scontents{ We have coded this in \LaTeX: $E=mc^2$.}
%
% \section{Test \texttt{\textbackslash getstored}}
%
% \getstored[1]{contents}\par
% \getstored[2]{contents}
%
% \section{Test \texttt{\textbackslash meaningsc}}
%
% \meaningsc[1]{contents}\par
%
% \meaningsc[2]{contents}
%
% \section{Test \texttt{\textbackslash typestored}}
%
% \typestored[1]{contents}
%
% \typestored*[2]{contents}
% \end{document}
% \end{VerbatimOut}
% Customization of \myenv*{verbatimsc} using the \mypkg{fancyvrb} and \mypkg{tcolorbox} package
% \textattachfile[color=0 0 1]{scexamp6.ltx}{\faFile*[regular]}.
% \lstinputlisting[language=scontents-doc,numbers=left]{scexamp6.ltx}
%
% \subsubsection*{Example 7}
%
% \begin{VerbatimOut}{scexamp7.ltx}
% \documentclass{article}
% \usepackage{scontents}
% \makeatletter
% \let\verbatimsc\@undefined
% \let\endverbatimsc\@undefined
% \makeatother
% \usepackage{xcolor}
% \usepackage{listings}
% \lstnewenvironment{verbatimsc}
%  {
%   \lstset{
%           basicstyle=\small\ttfamily,
%           breaklines=true,
%           columns=fullflexible,
%           language=[LaTeX]TeX,
%           numbers=left,
%           numbersep=1em,
%           numberstyle=\tiny\color{gray},
%           keywordstyle=\color{red}
%           }
%  }{}
% \setlength{\parindent}{0pt}
% \pagestyle{empty}
% \begin{document}
%
% \section{Test \texttt{\textbackslash begin\{scontents\}} whit \texttt{listings}}
% Test \verb+\begin{scontents}+ \par
%
% \begin{scontents}
% Using \verb+scontents+ env no \verb+[key=val]+, save in seq \verb+contents+ with index 1.\par
%
% Prove \lstinline[basicstyle=\ttfamily]| lstinline | and environment \verb+Verbatim*+
% \begin{verbatim}
%      verbatim  environment
% \end{verbatim}
% \end{scontents}
%
% \section{Test \texttt{\textbackslash Scontents*} whit \texttt{listings}}
%
% \Scontents*+ We have coded this in \lstinline[basicstyle=\ttfamily]|\LaTeX: $E=mc^2$|
% and more.+
%
% \section{Test \texttt{\textbackslash getstored}}
%
% \getstored[2]{contents}\par
%
% \getstored[1]{contents}
%
% \section{Test \texttt{\textbackslash typestored}}
%
% \typestored[1]{contents}
% \typestored*[2]{contents}
% \end{document}
% \end{VerbatimOut}
% Customization of \myenv*{verbatimsc} using the \mypkg{listings} package
% \textattachfile[color=0 0 1]{scexamp7.ltx}{\faFile*[regular]}.
% \lstinputlisting[language=scontents-doc,numbers=left]{scexamp7.ltx}
%
% \subsubsection*{Example 8}
%
% \begin{VerbatimOut}{scexamp8.ltx}
% \documentclass{article} % need shell-escape
% \usepackage{scontents}
% \makeatletter
% \let\verbatimsc\@undefined
% \let\endverbatimsc\@undefined
% \makeatother
% \usepackage{minted}
% \newminted{tex}{linenos}
% \newenvironment{verbatimsc}{\VerbatimEnvironment\begin{texcode}}{\end{texcode}}
% \pagestyle{empty}
% \begin{document}
% \section{Test \texttt{\textbackslash begin\{scontents\}} whit \texttt{minted}}
% Test \verb+\begin{scontents}+ \par
%
% \begin{scontents}
% Using \verb+scontents+ env no \verb+[key=val]+, save in seq \verb+contents+ with index 1.\par
%
% Prove new \Verb*{ new fvextra whit braces } and environment \verb+Verbatim*+
% \begin{verbatim}
%      verbatim environment
% \end{verbatim}
% \end{scontents}
%
% \section{Test \texttt{\textbackslash Scontents} whit \texttt{minted}}
%
% \Scontents{ We have coded \par this in \LaTeX: $E=mc^2$.}
%
% \section{Test \texttt{\textbackslash getstored}}
% \getstored[2]{contents}\par
% \getstored[1]{contents}
%
% \section{Test \texttt{\textbackslash typestored}}
% \typestored[1]{contents}
% \end{document}
% \end{VerbatimOut}
%
% Customization of \myenv*{verbatimsc} using the \mypkg{minted} package
% \textattachfile[color=0 0 1]{scexamp8.ltx}{\faFile*[regular]}.
% \lstinputlisting[language=scontents-doc,numbers=left]{scexamp8.ltx}
%
% \section{Change history}
%
% \label{sec:changes}
%
% \setlist[itemize,1]{label=\textendash, wide=0.5em, nosep, noitemsep, leftmargin=10pt}
% \newlength\descrwidth
% \settowidth{\descrwidth}{\textsf{v1.0, (ctan), 2019/07/30} }
%
% \begin{description}[font=\small\sffamily, wide=0pt, style=multiline, leftmargin=\descrwidth,  nosep, noitemsep]
% \item [\fileversion{} (ctan), \filedate]
%    \begin{itemize}
%        \item Restructuring of documentation.
%    \end{itemize}
% \item [v1.1 (ctan), 2019/08/12]
%    \begin{itemize}
%        \item Extension of documentation.
%        \item Replace \verb+\tex_endinput:D+ by \verb+\file_input_stop:+.
%    \end{itemize}
% \item [v1.0 (ctan), 2019/07/30]
%    \begin{itemize}
%        \item First public release,
%    \end{itemize}
% \end{description}
%
% \newpage
%
% \printindex[userdoc]
%
% \end{documentation}
%
% \newpage
%
% \StopEventually{^^A
% \newgeometry{top=0.5in,bottom=0.5in,left=1.0in,right=1in,footskip=0.2in,headsep=10pt}
% \addtocontents{toc}{\protect\setcounter{tocdepth}{2}}
% \cleardoublepage
% \phantomsection
% \printindex
% }
%
% \section{Implementation}
% \label{sec:Implementation}
%
% \addtocontents{toc}{\protect\setcounter{tocdepth}{0}}
%
% \begin{implementation}
%
% \iffalse
%<*package>
% \fi
% \subsection{Declaration of the package}
%
% First we set up the module name for \pkg{l3doc}:
%    \begin{macrocode}
%<@@=scontents>
%    \end{macrocode}
%
% Then, we can give the traditional declaration of a package written with
% \pkg{expl3} and the necessary packages for its operation.
%    \begin{macrocode}
\RequirePackage{filecontentsdef}[2019/04/20]
\RequirePackage{l3keys2e}
\RequirePackage{xparse}[2019/05/28]
\ProvidesExplPackage{scontents}{2019/08/26}{1.2}
  {Stores LaTeX contents in memory or files}
%    \end{macrocode}
% A check to make sure that \pkg{xparse} is not too old
%    \begin{macrocode}
\@ifpackagelater { xparse } { 2019/05/03 }
  { }
  {
    \PackageError { scontents } { Support~package~xparse~too~old }
      {
        You~need~to~update~your~installation~of~the~bundles~
        'l3kernel'~and~'l3packages'.\MessageBreak
        Loading~scontents~will~abort!
      }
    \file_input_stop:
  }
%    \end{macrocode}
%
% \subsection{Definition of common keys}
%
% We create some common keys that will be used by the options passed to
% the package as well as by the environments and commands defined.
%    \begin{macrocode}
\keys_define:nn { scontents }
  {
    store-env .tl_set:N         = \l_@@_name_seq_env_tl,
    store-env .initial:n        = contents,
    print-env .bool_set:N       = \l_@@_print_env_bool,
    print-env .initial:n        = false,
    store-cmd .tl_set:N         = \l_@@_name_seq_cmd_tl,
    store-cmd .initial:n        = contents,
    print-cmd .bool_set:N       = \l_@@_print_cmd_bool,
    print-cmd .initial:n        = false,
    print-all .meta:n           = { print-env = true , print-cmd = true },
    store-env .value_required:n = true,
    store-cmd .value_required:n = true,
    print-env .value_required:n = true,
    print-cmd .value_required:n = true,
    print-all .value_required:n = true
  }
%    \end{macrocode}
%
% We process the keys as options passed on to the package.
%%^^A Process options for pkg
%    \begin{macrocode}
\ProcessKeysOptions { scontents }
%    \end{macrocode}
%
% \subsection{Internal variables}
%
% Now we declare the internal variables we will use.
%
%%^^A Internal tl vars
% \begin{macro}{\l_@@_macro_tmp_tl,\l_@@_fname_out_tl,\l_@@_temp_tl}
%   \cs{l_@@_macro_tmp_tl} is a temporary token list to hold the contents
%   of the macro/environment, \cs{l_@@_fname_out_tl} is used as the name
%   of the output file, when there's one, and \cs{l_@@_temp_tl} is a
%   generic temporary token list.
%    \begin{macrocode}
\tl_new:N \l_@@_macro_tmp_tl
\tl_new:N \l_@@_fname_out_tl
\tl_new:N \l_@@_temp_tl
%    \end{macrocode}
% \end{macro}
%
%%^^A Internal bool vars
% \begin{macro}{\l_@@_typeverb_env_bool,\l_@@_writing_bool,
%      \l_@@_storing_bool}
%   The boolean \cs{l_@@_typeverb_env_bool} keeps track whether the
%   starred variant of the \cs{typestored} macro was used,
%   \cs{l_@@_writing_bool} if we should write to a file, and
%   \cs{l_@@_storing_bool} determines whether it is in write-only mode
%    when the |write-out| option is used.
%    \begin{macrocode}
\bool_new:N \l_@@_typeverb_env_bool
\bool_set_true:N  \l_@@_typevrb_env_bool
\bool_new:N \l_@@_writing_bool
\bool_set_false:N \l_@@_writing_bool
\bool_new:N \l_@@_storing_bool
\bool_set_true:N  \l_@@_storing_bool
%    \end{macrocode}
% \end{macro}
%
%%^^A Internal quarks
% \begin{macro}{\q_@@_stop,\q_@@_mark}
%   Some quarks used along the code as macro delimiters.
%    \begin{macrocode}
\quark_new:N \q_@@_stop
\quark_new:N \q_@@_mark
%    \end{macrocode}
% \end{macro}
%
% \begin{macro}{\g_@@_end_verbatimsc_tl}
%   A token list to match when ending verbatim environments.
%    \begin{macrocode}
\tl_new:N \g_@@_end_verbatimsc_tl
\tl_gset_rescan:Nnn
  \g_@@_end_verbatimsc_tl
  {
    \char_set_catcode_escape:N \|
    \char_set_catcode_other:N \\
    \char_set_catcode_other:N \{
    \char_set_catcode_other:N \}
  }
  { \end{verbatimsc} }
%    \end{macrocode}
% \end{macro}
%
% \subsection{Add keys for environment}
%
% We define a set of keys for environment \env{scontents}.
%%^^A Add keys to scontents environnment
%    \begin{macrocode}
\keys_define:nn { scontents }
  {
    write-env .code:n           = {
                                    \bool_set_true:N \l_@@_writing_bool
                                    \tl_set:Nn \l_@@_fname_out_tl {#1}
                                  },
    write-out .code:n           = {
                                    \bool_set_false:N \l_@@_storing_bool
                                    \bool_set_true:N  \l_@@_writing_bool
                                    \tl_set:Nn \l_@@_fname_out_tl {#1}
                                  },
    write-env .value_required:n = true,
    write-out .value_required:n = true
  }
%    \end{macrocode}
%
% \subsection{Define keys for command}
%
% A sub/keys for command |\Scontents| and |\Scontents*|
%
%%^^A A sub/keys for command |\Scontents|
%    \begin{macrocode}
\keys_define:nn { scontents / Scontents }
  {
    print-cmd .meta:nn = { scontents } { print-cmd = #1 },
    store-cmd .meta:nn = { scontents } { store-cmd = #1 }
  }
%    \end{macrocode}
%
% \subsection{Programming of the sequences}
%
%%^^A Append content to seq
% \begin{macro}{\@@_append_contents:nn,\@@_getfrom_seq:nn}
%   The storage of the package is done using |seq| variables.  Here we
%   set up the macros that will manage the variables.
%
%   \cs{@@_append_contents:nn} creates a seq variable if one didn't
%   exist and appends the contents in the argument to the right of the
%   sequence.  \cs{@@_getfrom_seq:nn} retrieves the saved item from the
%   sequence.
%    \begin{macrocode}
\cs_new_protected:Npn \@@_append_contents:nn #1#2
  {
    \tl_if_blank:nT {#1}
      { \msg_error:nn { scontents } { empty-store-cmd } }
    \seq_if_exist:cF { g_@@_seq_name_#1_seq }
      { \seq_new:c { g_@@_seq_name_#1_seq } }
    \seq_gput_right:cn { g_@@_seq_name_#1_seq } {#2}
  }
\cs_generate_variant:Nn \@@_append_contents:nn { Vx }
\cs_new:Npn \@@_getfrom_seq:nn #1#2
  { \seq_item:cn { g_@@_seq_name_#2_seq } {#1} }
%    \end{macrocode}
% \end{macro}
%
% \subsection{Construction of environment scontents}
%
% We define the environment \env{scontents}, next to the system |key=val|.
% The environment is divided into three parts.  This implementation is taken
% from answer by Enrico Gregorio in \url{https://tex.stackexchange.com/a/487746/7832}.
%
%%^^A Define |scontents| whit [key=val] (delaying)
% \begin{macro}{scontents}
%   This is the main environment used in the document.
%    \begin{macrocode}
\ProvideDocumentEnvironment { scontents } { }
  {
    \char_set_catcode_active:N \^^M
    \@@_start_environment:w
  }
  {
    \@@_stop_environment:
    \@@_atend_environment:
  }
%    \end{macrocode}
% \end{macro}
%
% \subsubsection{The environment itself}
%
% The environment itself
%
%%^^A First \cs{@@_start_environment:w}
% \begin{macro}{\@@_start_environment:w,\@@_stop_environment:}
%   Here we make |^^M| an active character so that the end of line can
%   be ``seen'' to be used as a delimiter.  First we check if the line
%   directly after |\begin{scontents}| contains an optional argument
%   enclosed in |[...]|, or other tokens.  The trailing tokens are
%   treated as junk and an error is raised.  The
%   \cs{@@_environment_inline:w} macro checks for those cases.
%    \begin{macrocode}
\group_begin:
  \char_set_catcode_active:N \^^M
  \cs_new_protected:Npn \@@_start_environment:w #1 ^^M
    {
      \@@_environment_inline:w #1 \q_@@_mark
      \group_begin:
        \bool_if:NTF \l_@@_writing_bool
          {
            \use:c { filecontentsdef* } { \l_@@_fname_out_tl }
                                        { \l_@@_macro_tmp_tl } ^^M
          }
          { \filecontentsdefmacro { \l_@@_macro_tmp_tl } ^^M }
    }
  \cs_new_protected:Npn \@@_stop_environment:
    {
        \bool_if:NTF \l_@@_writing_bool
          { \endfilecontentsdef }
          { \endfilecontentsdefmacro }
      \group_end:
    }
\group_end:
%    \end{macrocode}
% \end{macro}
%
% \subsubsection{key val for environment}
%
% Define a |key=val| for environment \env{scontents}
%
%%^^A key val for environment scontents
% \begin{macro}{\@@_environment_inline:w,\@@_environment_keys:w,
%      \@@_environment_junk:nw,\@@_environment_junk:xw}
%   The macro \cs{@@_environment_inline:w} is called from the
%   \env{scontents} environment with the tokens following the
%   |\begin{scontents}|.  If the immediate next token (ignoring spaces)
%   is a |[|, then we look for an optional argument delimited by a |]|.
%   All the remaining tokens are treated as junk and an error is raised
%   if they are non-blank.
%    \begin{macrocode}
\cs_new_protected:Npn \@@_environment_inline:w
  {
    \peek_charcode_ignore_spaces:NTF [ % ]
      { \@@_environment_keys:w }
      {
        \@@_environment_junk:xw
          { after~\c_backslash_str begin{scontents} }
      }
  }
\cs_new_protected:Npn \@@_environment_keys:w [ #1 ]
  {
    \keys_set_known:nn { scontents } {#1}
    \@@_environment_junk:xw
      { after~optional~argument~to~\c_backslash_str begin{scontents} }
  }
\cs_new_protected:Npn \@@_environment_junk:nw #1 #2 \q_@@_mark
  {
    \tl_if_blank:nF {#2}
      { \msg_error:nnnn { scontents } { junk-after-begin } {#1} {#2} }
  }
\cs_generate_variant:Nn \@@_environment_junk:nw { x }
%    \end{macrocode}
% \end{macro}
%
% \subsubsection{Recording of the content in the sequence}
%
% \begin{macro}{\@@_atend_environment:,\@@_stored_to_seq:}
%   Finishes the environment by optionally calling \cs{@@_stored_to_seq:}
%   and then clearing the temporary token list.
%
%   The \cs{@@_stored_to_seq:} function replaces a carriage return
%   (\textsc{ascii} 13) by a new line character (\textsc{ascii} 10) for
%   optionally logging the contents of the current \env{scontents}
%   environment.
%
%^^A TODO: Perhaps replace the regex by a delimited macro to improve
%^^A       performance... Or temporarily redefine ^^M to mean ^^J...
%^^A Reply: Or maybe it is better to preserve the end of line with
%^^A       \tex_newlinechar:D = 13 \scan_stop:
%    \begin{macrocode}
\cs_new_protected:Npn \@@_atend_environment:
  {
    \bool_if:NT \l_@@_storing_bool
      {
        \@@_stored_to_seq:
        \bool_if:NT \l_@@_print_env_bool
          { \@@_getfrom_seq:nn { -1 } { \l_@@_name_seq_env_tl } }
      }
    \tl_clear:N \l_@@_macro_tmp_tl
  }
\cs_gset_protected:Npn \@@_stored_to_seq:
  {
    \regex_replace_all:nnN { \^^M } { \^^J } \l_@@_macro_tmp_tl
    \tl_log:N \l_@@_macro_tmp_tl
    \@@_append_contents:Vx \l_@@_name_seq_env_tl
      { \exp_not:N \tex_scantokens:D { \tl_use:N \l_@@_macro_tmp_tl } }
  }
%    \end{macrocode}
% \end{macro}
%
% \subsection{The \cs{Scontents} command}
%
% User command to stored content, adapted from
% \url{https://tex.stackexchange.com/a/500281/7832}.
%
%%^^A User command to stored content
%
% \begin{macro}{\Scontents,\@@_norm:n,\@@_verb:w}
%   The \cs{Scontents} macro starts by parsing an optional argument and
%   then delegates to \cs{@@_verb:w} or \cs{@@_norm:n} depending whether
%   a star argument is present.
%
%   \cs{@@_norm:n} grabs a normal argument, adds it to the |seq| varaible,
%   and optionally prints it.
%
%   \cs{@@_verb:w} grabs a verbatim argument using \pkg{xparse}'s |v|
%   argument parser.
%
%^^A NOTE: Any particular reason why \keys_set_known:nn instead of just
%^^A       \keys_set:nn? The latter reports if an unknown key is used...
%^^A Reply: The redirection of error messages has never worked well for me
%^^A       perhaps it would be better to implement it.
%    \begin{macrocode}
\ProvideDocumentCommand { \Scontents }{ !s !O{} }
  {
    \group_begin:
      \IfNoValueF {#2}
        { \keys_set_known:nn { scontents / Scontents } {#2} }
      \IfBooleanTF{#1}
        { \@@_verb:w }
        { \@@_norm:n }
  }
\cs_new_protected:Npn \@@_norm:n #1
  {
     \exp_args:NV \@@_append_contents:nn \l_@@_name_seq_cmd_tl {#1}
     \bool_if:NT \l_@@_print_cmd_bool
       { \@@_getfrom_seq:nn { -1 } { \l_@@_name_seq_cmd_tl } }
   \group_end:
  }
\NewDocumentCommand { \@@_verb:w } { +v }
  {
      \tl_set:Nn \l_@@_temp_tl {#1}
      \regex_replace_all:nnN { \^^M } { \^^J } \l_@@_temp_tl
      \tl_log:N \l_@@_temp_tl
      \exp_args:NVx \@@_append_contents:nn \l_@@_name_seq_cmd_tl
        { \exp_not:N \tex_scantokens:D { \tl_use:N \l_@@_temp_tl } }
      \bool_if:NT \l_@@_print_cmd_bool
        { \@@_getfrom_seq:nn { -1 } { \l_@@_name_seq_cmd_tl } }
    \group_end:
  }
%    \end{macrocode}
% \end{macro}
%
% \subsection{The command \cs{getstored}}
%
%
% \begin{macro}{\getstored}
%   User command \cs{getstored} to extract stored content in |seq|
%   (robust).
%    \begin{macrocode}
\ProvideDocumentCommand { \getstored } { O{1} m }
  { \@@_getfrom_seq:nn {#1} {#2} }
%    \end{macrocode}
% \end{macro}
%
% \subsection{The \cs{typestored} command}
%
% This implementation is an adaptation taken from answer by Phelype Oleinik
% in (\url{https://tex.stackexchange.com/a/497651/7832}).
%
% \begin{macro}{\typestored,\@@_fcdef_print:N,\@@_xverb:w,verbatimsc}
%   The \cs{typestored} commands fetches a buffer from memory, prints it
%   to the log file,
%^^A NOTE: Perhaps this logging should be optional, with a key to switch
%^^A       it on and off.
%^^A Reply:I used * because it's easier to associate it with \Scontents*
%   and then calls \pkg{filecontentsdef}'s \tn{filecontentsdef@get} macro
%   to read the contents of the token list and pass them to
%   \cs{@@_fcdef_print:N}.
%    \begin{macrocode}
\ProvideDocumentCommand { \typestored } { s O{1} m }
  {
    \group_begin:
      \tl_set:Nx \l_@@_temp_tl { \@@_getfrom_seq:nn {#2} {#3} }
      \tl_log:N \l_@@_temp_tl
      \IfBooleanTF {#1}
        { \bool_set_false:N \l_@@_typeverb_env_bool }
        { \bool_set_true:N  \l_@@_typeverb_env_bool }
      \use:c { filecontentsdef@get } \@@_fcdef_print:N \l_@@_temp_tl
    \group_end:
  }
%    \end{macrocode}
%   The \cs{@@_fcdef_print:N} macro is defined with active carriage return
%   (\textsc{ascii} 13) characters to mimick an actual verbatim environment
%   ``on the loose''. The contents of the environment are placed in a
%   \env{verbatimsc} environment and rescanned using \tn{scantokens}.
%    \begin{macrocode}
\group_begin:
 \char_set_catcode_active:N \^^M
 \cs_new_protected:Npn \@@_fcdef_print:N #1
    {
      \tl_if_blank:VT #1
        { \msg_error:nnn { scontents } { empty-variable } {#1} }
      \cs_set_eq:NN \@@_fcdef_saved_EOL: ^^M
      \cs_set_eq:NN ^^M \scan_stop:
      \use:x
        {
          \exp_not:N \tex_scantokens:D
            {
              \exp_not:N \begin{verbatimsc} ^^M
              \@@_strip_scantokens:N #1
              \bool_if:NF \l_@@_typeverb_env_bool { ^^M }
              \g_@@_end_verbatimsc_tl
            }
        }
      \cs_set_eq:NN ^^M \@@_fcdef_saved_EOL:
    }
\group_end:
%    \end{macrocode}
%   Finally, the \env{verbatimsc} environment is defined.
%^^A An idea taken from spverbatim
%    \begin{macrocode}
\use:x
  { \cs_gset_protected:Npn \exp_not:N \@@_xverb:w ##1 \g_@@_end_verbatimsc_tl }
      { #1 \end{verbatimsc} }
\ProvideDocumentEnvironment { verbatimsc } { }
  {
    \cs_set_eq:cN { @xverbatim } \@@_xverb:w
    \verbatim
  }
  { }
%    \end{macrocode}
% \end{macro}
%
% \begin{macro}{
%     \@@_strip_scantokens:N,
%     \@@_strip_scantokens:n,
%     \@@_if_scantokens:Nw,
%   }
%   The \cs{@@_strip_scantokens:n} (\cs{@@_strip_scantokens:N}) macro
%   takes a token list (variable) as argument and examines it. If the
%   argument token list is \emph{exactly} |\scantokens{|\meta{stuff}|}|,
%   then the function returns \meta{stuff}, otherwise it returns the
%   input token list without change. The token list is wrapped in
%   \cs{exp_not:n} to avoid further expansion.
%    \begin{macrocode}
\cs_new:Npn \@@_strip_scantokens:N #1
  { \exp_args:NV \@@_strip_scantokens:n #1 }
\cs_new:Npn \@@_strip_scantokens:n #1
  {
    \tl_if_head_is_N_type:nTF {#1}
      {
        \@@_if_scantokens:NwTF #1 \q_@@_mark
          {
            \exp_args:No \tl_if_single_token:nTF { \use_none:nn #1 ? }
              { \exp_not:o { \use_ii:nn #1 } }
              { \exp_not:n {#1} }
          }
          { \exp_not:n {#1} }
      }
      { \exp_not:n {#1} }
  }
\prg_new_conditional:Npnn \@@_if_scantokens:Nw #1#2 \q_@@_mark { TF }
  {
    \token_if_eq_meaning:NNTF \tex_scantokens:D #1
      { \prg_return_true: }
      { \prg_return_false: }
  }
%    \end{macrocode}
% \end{macro}
%
% \subsection{The command \cs{setupsc}}
%
% User command \cs{setupsc} to setup module.
%%^^A User command to setup module
% \begin{macro}{\setupsc}
%   A user-level wrapper for \cs{keys_set:nn}|{ scontents }|
%    \begin{macrocode}
\ProvideDocumentCommand { \setupsc } { m }
  { \keys_set:nn { scontents } {#1} }
%    \end{macrocode}
% \end{macro}
%
% \subsection{The command meaningsc}
%
% \begin{macro}{\meaningsc}
%   User command \cs{meaningsc} to see content stored in seq.
%^^A NOTE: Perhaps replace \cs{regex_replace_all:nnN} by something more
%^^A       efficient...
%^^A Reply: I've only used it to avoid ΩΩ ...
%    \begin{macrocode}
\ProvideDocumentCommand { \meaningsc } { O{1} m }
  {
    \group_begin:
      \tl_set:Nx \l_@@_temp_tl { \@@_getfrom_seq:nn {#1} {#2} }
      \tl_log:N \l_@@_temp_tl
      \tl_set:Nx \l_@@_temp_tl { \@@_strip_scantokens:N \l_@@_temp_tl }
      \regex_replace_all:nnN { \v{1,} } {   } \l_@@_temp_tl
      \ttfamily
      \cs_replacement_spec:N \l_@@_temp_tl
    \group_end:
  }
%    \end{macrocode}
% \end{macro}
%
% \subsection{The command \cs{countsc}}
%
% \begin{macro}{\countsc}
%   User command \cs{countsc} to count number of contents stored in seq.
%    \begin{macrocode}
\ProvideExpandableDocumentCommand { \countsc } { m }
  { \seq_count:c { g_@@_seq_name_#1_seq } }
%    \end{macrocode}
% \end{macro}
%
% \subsection{The command \cs{cleanseqsc}}
%
% \begin{macro}{\cleanseqsc}
%   A user command \cs{cleanseqsc} to clear (remove) a defined seq
%    \begin{macrocode}
\ProvideExpandableDocumentCommand { \cleanseqsc } { m }
  { \seq_clear_new:c { g_@@_seq_name_#1_seq } }
%    \end{macrocode}
% \end{macro}
%
% \subsection{Messages}
% Messages used throughout the package.
%%^^A Messages
%
%    \begin{macrocode}
\msg_new:nnn { scontents } { junk-after-begin }
  {
    Junk~characters~#1~\msg_line_context: :
    \\ \\
    #2
  }
\msg_new:nnn { scontents } { empty-stored-content }
  { Empty~value~for~key~`getstored'~\msg_line_context:. }
\msg_new:nnn { scontents } { empty-variable }
 { Variable~`#1'~empty~\msg_line_context:. }
%    \end{macrocode}
%
% \subsection{Finish package}
% Finish package
%%^^A Finish package
%
%    \begin{macrocode}
\file_input_stop:
%    \end{macrocode}
% \iffalse
%</package>
% \fi
%
% \end{implementation}
% \Finale
